\documentclass[a4paper,10pt]{article}
\usepackage[utf8x]{inputenc}
\usepackage[italian]{babel}
\usepackage{amsmath}

\title{Documento di \LaTeX}
\author{Giovanni Canarecci}
\date{\today}

\begin{document}

\maketitle
\section{\textbf{Le formule matematiche}}
Fondamentale è riuscire e scrivere formule matematiche come $(x+y) = 1$ utilizzando il \LaTeX. 
Studiamo la seguente situazione

 $$
\begin{cases}
-x^2,\qquad &if\quad x<0;\\
\alpha + x,\qquad &if\quad0\leq x\leq1;\\
x^2\qquad &otherwhise.
\end{cases}
 $$

\noindent
Ora proviamo a scrivere un integrale, equazione \eqref{eq}\\

\begin{equation}
\label{eq}
\int_{0}^{\pi}\sin xdx = 2
 \end{equation}

Continuando

\begin{align*}
 X_t&=\alpha^{'}(t)+v\alpha^{``}\\
 X_v&=\alpha^{'}(t)\\
\Rightarrow &X_{t}\wedge X_{v}=(\alpha{'}(t)+v\alpha^{''}(t))\wedge\alpha^{'}(t)=\\
&=v\alpha^{''}(t)\wedge \alpha^{'}(t)\neq 0 \Leftrightarrow v\neq 0;
\end{align*}

Alcuni esempi con le parentesi:
$$
\binom{\frac{n^{2}+1}{2}}{n+1}
$$

invece passando ad una matrice
\begin{align*}
\mathbf{A}=\begin{pmatrix}
            a+b+c&uv\\
            a+b&u+v
           \end{pmatrix}
\begin{vmatrix}
 30&7\\
 3&17
\end{vmatrix}
\\
\int_{-\infty}^{+\infty}e^{-x^{2}}dx= \sqrt{\pi}
\end{align*}

\textbf{Apici e pedici}
$$
a_1,a_{i_{1}},a^2,a^{b^{c}},a^{i_{1}},a_{i}+1,a_1^2
$$

e un pò di radici

$$
1+\sqrt{1+\frac{1}{2}\sqrt{1+\frac{1}{3}\sqrt{1+\frac{1}{4}}}}
$$

Sia$ A=\{x \lvert \quad for\quad x \quad large\}$, la frazione

$$
\frac{\sqrt{\mu(i)^{\frac{3}{2}}(i^{2}-1)}}{\sqrt[3]{\rho(i)-2}+\sqrt[3]{\rho(i)-1}}
$$




\end{document}
